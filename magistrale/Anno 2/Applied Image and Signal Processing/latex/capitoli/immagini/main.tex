\chapter{Le immagini digitali}
\section{Definizione di immagine}
Useremo il Teorema del campionamento per applicarlo al concetto di immagine.
\begin{definition}
    Un’immagine è una rappresentazione grafica di valori numerici.
\end{definition}
In dettaglio un’immagine è una funzione bi-dimensionale $f(x,y)$, dove le variabili (spaziali) $x$ e $y$ sono valori reali che definiscono la posizione dei punti nell’immagine e $f(x,y)$ e in genere un valore reale che definisce l’intensità dell’immagine nel punto $(x,y)$.
\\\\Il punto che andiamo a definire con le coordinare $x$, $y$ definisce il punto di grigio, al quale appartiene una data intensità.
\\\\Tutti i colori al calcolatore possono essere scoomposti mediante combinazioni di \textbf{RGB}.
\\Dove:
\begin{center}
    $R = f_1$, $G = f_2$, $B = f_3$
\end{center}
\section{Rappresentazione di un’immagine}
La funzione f che rappresenta l’immagine può essere a valori in R,
in $R^2$ o in $R^3$, a seconda del tipo di immagine.
\begin{itemize}
    \item \textbf{Immagine in scala di grigi:} $f:R^2 \rightarrow R$ (funzione scalare)
    \item \textbf{Immagine a colori:} $f:R^2 \rightarrow R^3$ (funzione vettoriale)
\end{itemize}
Ovvero:
\begin{center}
    $f(x,y) = [f_1(x,y), f_2(x,y), f_3(x,y)]$
\end{center}
dove le componenti $f_i$, $i = 1,2,3$ si dicono canali.
\\\\Se vogliamo rappresentare una scena in movimento, ottenendo
ciò e un’\textbf{immagine dinamica}, è necessario introdurre una terza
variabile, quella \textbf{temporale} ($t$), per cui si lavora con una funzione
$f: R^3 \rightarrow R^3$.
\begin{center}
    $f(x,y,t) = [f_1(x,y,t), f_2(x ,y,t), f_3(x,y,t)].$
\end{center}
Nelle immagini \textbf{Analogiche} conosco l'intensità di ogni livello di grigio in ogni punto.
Le immagini mostrate al calcolatore invece vanno \textbf{DISCRETIZZATE!}
\section{Discretizzazione}
Se si vuole utilizzare un calcolatore elettronico per lo studio di un
segnale, è necessario \textbf{discretizzare} la funzione $s(t)$ che rappresenta
il segnale. Infatti un calcolatore elettronico è in grado di trattare
solo segnali discreti, cioè successioni di campioni i cui valori sono
rappresentati con precisione finita.
\\Se si lavora con un segnale continuo $s(t)$, per implementarne lo
studio al calcolatore è necessario passare ad un opportuno segnale
discreto.
\begin{center}
    Ciò avviene utilizzando il procedimento di \textbf{campionamento}, che consiste nel discretizzare la variabile temporale $t$.
\end{center}
Inoltre, è anche necessario discretizzare i valori che la funzione $s(t)$
assume (\textbf{quantizzazione}).
\\\\Nel caso delle immagini appliccare i processi di:
\begin{center}
    \textbf{campionamento + quantizzazione}
\end{center}
significa passare da un'immagine \textbf{analogica} ad un'immagine \textbf{digitale}
\section{Campionamento di un segnale}
Il campionamento di un segnale può essere fatto in 2 diversi modi:
\begin{enumerate}
    \item \textbf{Nel tempo:} Il campionamento di un segnale si ottiene prelevando
          i valori che il segnale assume soltanto in istanti temporali fissati, in genere individuati tramite una funzione periodica \textbf{(funzione campionante)}.
          La successione dei valori campionati di s fornisce una rappresentazione \textbf{discreta} (nel tempo) di $s(t)$.
    \item \textbf{Nello spazio:} Un’immagine può essere vista come una funzione $f(x,y,t)$ dello spazio e del tempo e dunque è necessario discretizzare anche le variabili spaziali.
          \\Si ottiene in questo modo una matrice a tre dimensioni, delle quali due sono spaziali ed una è temporale.
\end{enumerate}
\section{Funzione Campionante}
In genere, si assume che il campionamento sia \textbf{uniforme}, sia dal punto di vista spaziale che temporale, ovvero che la funzione campionante sia periodica di periodo costante.
\\\\Fissiamo gli intervalli di campionamento $\Delta x$ , $\Delta y$, $\Delta t$ appropriati (dal Teorema Sampling e dalla teoria di Nyquist), ovvero la
distanza tra due campioni successivi lungo le coordinate $x$,$y$ e $t$.
\\\\Indichiamo con $N$,$M$,$T$ le dimensioni della matrice dei valori campionati dell’immagine.
\\\\Andiamo a definire \textbf{la funzione campionante} come:
\begin{center}
    $s_c(x,y,t) = \sum_{j=1}^{M} \sum_{k=1}^{N}\sum_{h=1}^{T} \delta (x-j, y - k \Delta y, t - h  \Delta t )$
\end{center}
L'immagine \textbf{campionata diventa}:
\begin{center}
    $s_c(x,y,t) = f(x,y,t)s_c(x,y,t)$
\end{center}
\begin{center}
    $= f(x,y,t) \sum_{j=1}^{M} \sum_{k=1}^{N}\sum_{h=1}^{T} \delta (x-j, y - k \Delta y, t - h  \Delta t )$
\end{center}
Lo scopo della funzione campionante sc(x , y, t) è di prelevare i
valori campionati dal segnale continuo di partenza e pertanto ha
un caratteristico andamento \textbf{pulsante}.ù
\begin{itemize}
    \item Il segnale \textbf{non va mai letto}

          quando x cade nel nodo della funzione in quanto non si sarebbe in grado di leggerlo.

    \item Il segnale \textbf{va letto}

          soltanto in $\frac{j}{w}$ ovvero la funzione campionante parellela ai campioni.
\end{itemize}
%TODO: Inserire foto

\section{Quantizzazione}
Per ottenere una completa discretizzazione di un’immagine è
necessario discretizzare, oltre al dominio, anche l’insieme immagine
(insieme dei valori).
\begin{definition}
    Si definisce \textbf{quantizzazione} il procedimento di discretizzazione dei
    valori della funzione che rappresenta un’immagine, cioè il passaggio
    da valori continui a valori discreti.
\end{definition}
Per le immagini a toni di grigio si parla di \textbf{grey level quantization},
mentre per le immagini a colori si parla di \textbf{color depth}, in
riferimento al numero di bit utilizzati per ciascun canale di colore
(8, 16, 24, 32 bit).
\begin{itemize}
    \item \textbf{Esempio:}

          Nelle immagini mediche formato \textbf{DOICOM} gli ultimi 4 bit dell'immagine
          sono riservati ad informazioni personali che servono ad identificare il paziente che ha sostenuto l'esame.
\end{itemize}
\section{Immagine Digitale}
Tramite il campionamento e la quantizzazione è possibile definire
un’immagine digitale come segue:
\begin{definition}
    Una immagine digitale è una rappresentazione di matrici di
    elementi immagine, detti anche pixel (pixel= picture elements).
\end{definition}
\begin{itemize}
    \item Il \textbf{pixel} costituisce la componente elementare della matrice,
          dove gli indici di riga e colonna indicano i valori delle due
          variabili spaziali, cioè la posizione di un punto nell’immagine.
    \item Ogni elemento della matrice contiene i valori che
          rappresentano l’intensità dei corrispondenti punti
          nell’immagine, anche detta \textbf{luminanza}.
\end{itemize}
\section{Teorema Campionamento Immagini}
L'Immagine campionata è rappresentata tramite la seguente formula:
\begin{center}
    $s_c(x,y) = f(x,y)s_c(x,y)=f(x,y)\sum_{j=-\infty}^{+\infty} \sum_{k=-\infty}^{+\infty} \delta (x-j \Delta x, y-k \Delta y)$
\end{center}
dove $s_c(x,y)$ è \textbf{la funzione campionante}.
\\\\Si può provare che c’è una relazione tra $\hat{f}_c$ e $\hat{f}$. Per questo è importante assumere che lo spettro del segnale f sia \textbf{a banda limitata}, cioè:
\begin{center}
    $\hat{f}(w_x,w_y)=0$ per $|w_x| > \bar{w}_x$ e $|w_y| > \bar{w}_y$
\end{center}
dove $\bar{ω_x}$ e $\bar{ω_y}$ definiscono la banda rettangolare dell’immagine.
\\Così lo spettro dell’immagine campionata consiste nello spettro
dell’immagine continua infinitamente ripetuta nel piano delle
frequenze, in una griglia di risoluzione ($\frac{2\pi}{\Delta x}, \frac{2 \pi}{\Delta y}$), dove:
\begin{center}
    ($\frac{2\pi}{\Delta x}, \frac{2 \pi}{\Delta y}$) = $(w_{xe}, w_{ye})$
\end{center}
sono le \textbf{frequenze Sampling}.
\\\\Per ricostruire esattamente un segnale campionato, la frequenza
del campionamento non deve essere inferiore ad una \textbf{frequenza
    minima (ovvero frequenza sampling)}, che corrisponde ad un valore
massimo per ciascuno degli intervalli $\Delta x$ , $\Delta y$.
\\\\Tale valore minimo deve essere almeno pari al doppio della banda
massima di $f$ , cioè:
\begin{center}
    $w_{xe} >= 2 \bar{w_x}$ e $w_{ye} >= 2 \bar{w_y}$
\end{center}
Se nella (1) vale l’uguaglianza, allora si dice che l’immagine è
\textbf{campionata alla sua frequenza di Nyquist.}
\\\\Se $\Delta x$ e $\Delta y$ sono più piccoli del richiesto criterio di Nyquist,
l’immagine risulta sovracampionata \textbf{(oversampling)}.
Nel caso contrario, l’immagine non può essere ricostruita
esattamente: si parla di sottocampionamento \textbf{(undersampling)} e si
presenta un fenomeno di distorsione detto \textbf{aliasing.}
\begin{itemize}
    \item \textbf{Osservazione:}

          il valore minimo è un valore puramente teorico.
          \\Nella pratica, non potendo in generale determinare con precisione la banda massima del segnale, si utilizzano frequenze di campionamento più alte.
          \\Spesso si campiona con una frequenza pari a 4 volte quella misurata.
\end{itemize}
\begin{theorem}
    Sia $f(x,y)$ una immagine a banda limitata e ad energia finita,
    soddisfacente quindi la condizione $\hat{f}(w_x,w_y) = 0$, per $| w_x | > \bar{ω_x}$ e
    $| w_y | > \bar{w_y}$; se $f$ è uniformemente campionata in una griglia rettangolare
    con intervalli spaziali $\Delta x$, $\Delta y$, allora la $f$ può essere ricostruita dai suoi
    valori campione $f(j \Delta x, k \Delta y)$ a condizione che l’ordine di campionamento
    sia più grande dell’ordine di Nyquist, cioè $w_{xe} >= 2 \bar{w_x}$, $w_{ye} >= 2 \bar{w_y}$.
    Inoltre,l’immagine ricostruita è data dalla seguente formula di interpolazione:
\end{theorem}
\begin{center}
    $f(x,y) = \sum_{j=-\infty}^{+\infty} \sum_{k=-\infty}^{+\infty} f(j \Delta x, k \Delta y) (\frac{\sin(xw_{xe}-j)\pi}{(xw_{xe}-j)\pi})  (\frac{\sin(yw_{ye}-k)\pi}{(yw_{ye}-k)\pi})$
\end{center}
\section{L'aliasing}
Per ricostruire esattamente una immagine, è necessario limitare in
banda l’immagine che deve essere campionata, campionando all’ordine di campionamento di Nyquist o più grande e
interpolando appropriatamente i valori immagine.
\\\\Se c’è sovrapposizione di spettri, risultante dal
sottocampionamento, vuol dire che componenti spettrali spurie
sono state introdotte nel processo di ricostruzione. L’effetto che si
ottiene è chiamato aliasing.
%TODO: Inserire foto appunti fatti a mano 

Quindi l'aliasing è la presenza di componenti spettrali (frequenze)
indesiderate nella ricostruzione dell’immagine, componenti che non
erano presenti quando l’immagine originale era stata campionata.

%TODO Inserire immagini

L’aliasing deriva dal sottocampionamento e causa perdita di
risoluzione dell’immagine campionata (effetto scacchiera).

%TODO Inserire Immagini

\begin{itemize}
    \item Per prevenire aliasing di queste componenti, è possibile
          filtrarle via (eliminarle) prima di campionare il segnale.
          Eliminare certe frequenze e lasciare passare le basse frequenze,
          è una operazione nota come \textbf{filtraggio passa-basso}.
    \item Ogni attenuazione relativa a questo processo di filtraggio
          rappresenta una perdita di risoluzione dell’immagine
          campionata.
    \item Come risultato, mentre da un lato c’è una perdita della
          risoluzione dell’immagine campionata, dall’altro c’è una
          attenuazione dell’aliasing error.
    \item \textbf{Effetto Moirè:} ovvero la distorsione visiva che si manifesta quando due
          griglie si sovrappongono
          %TODO: Inserire foto
\end{itemize}
\section{La risoluzione}
Il campionamento e la quantizzazione determinano la \textbf{risoluzione}
dell’immagine.
\\\\La \textbf{risoluzione} di un segnale è un indice del grado di qualità
dell’immagine: misura il grado di oggetti distinguibili
nell’immagine.
\begin{itemize}
    \item \textbf{Risoluzione spaziale}

          Indica la densità dei campioni, ovvero è data dal numero di campioni per unità di area.
          \\\\Spesso è espressa come numero di pixel nell’unità di lunghezza
          e viene misurata in pixel per pollice (ppi).
          \\Un’immagine ad alta risoluzione contiene più pixel di una delle
          stesse dimensioni con una risoluzione inferiore, quindi è in
          grado di riprodurre un maggior numero di dettagli. Un’elevata
          risoluzione comporta tuttavia un aumento considerevole delle
          dimensioni (quantità di dati) dell’immagine.
          \begin{itemize}
              \item \textbf{Esempio:}

                    Un’immagine di 1cm x 1cm con una risoluzione di 72 ppi
                    contiene 5184 pixel (72 x 72). La stessa immagine di 1 cm x 1 cm a 300 ppi conterrebbe 90.000 pixel.
          \end{itemize}
    \item \textbf{Risoluzione spettrale}

          Indica la banda passante del sensore.

    \item \textbf{Risoluzione radiometrica}

          indica il numero di livelli di quantizzazione.

    \item \textbf{Risoluzione temporale}

          indica la frequenza di acquisizione dei frames di un’immagine in movimento.
\end{itemize}
\section{Alterazioni della risoluzione}
Alterando i vari tipi di risoluzione, l’immagine presenterà di volta in
volta un diverso tipo di distorsione.
\\ \textbf{Immagine originale:}
%TODO Inserire immagine
\begin{trivlist}
    \item \textbf{Risoluzione spaziale:} diminuendo la risoluzione spaziale
    (nell’esempio di un quarto) si ottiene il tipico effetto
    ”quadrettato”, detto anche a scacchiera, dovuto all’aliasing.
    %TODO Inserire immagine
    \item \textbf{Risoluzione spettrale:} Diminuendo la banda passante del
    sensore di acquisizione dell’immagine si ottiene un’immagine
    più ”sfocata”, in quanto i dettagli ad alta frequenza spaziale
    vanno persi.
    %TODO Inserie Immagine
    \item \textbf{Risoluzione radiometrica:} Diminuendo la profondità di
    colore, si distinguono in maniera più marcata i passaggi da un
    colore ad un altro; essi risultano pertanto sempre più
    accentuati e meno graduali, fino a produrre dei ”falsi contorni”
    %TODO Inserire Immagine
\end{trivlist}

