\chapter{Le immagini digitali}
\section{Definizione di immagine}
Useremo il Teorema del campionamento per applicarlo al concetto di immagine.
\begin{definition}
    Un’immagine è una rappresentazione grafica di valori numerici.
\end{definition}
In dettaglio un’immagine è una funzione bi-dimensionale $f(x,y)$, dove le variabili (spaziali) $x$ e $y$ sono valori reali che definiscono la posizione dei punti nell’immagine e $f(x,y)$ e in genere un valore reale che definisce l’intensità dell’immagine nel punto $(x,y)$.
\\\\Il punto che andiamo a definire con le coordinare $x$, $y$ definisce il punto di grigio, al quale appartiene una data intensità.
\section{Rappresentazione di un’immagine}
La funzione f che rappresenta l’immagine può essere a valori in R,
in $R^2$ o in $R^3$, a seconda del tipo di immagine.
\begin{itemize}
    \item \textbf{Immagine in scala di grigi:} $f:R^2 \rightarrow R$ (funzione scalare)
    \item \textbf{Immagine a colori:} $f:R^2 \rightarrow R^3$ (funzione vettoriale)
\end{itemize}
Ovvero:
\begin{center}
    $f(x,y) = [f_1(x,y), f_2(x,y), f_3(x,y)]$
\end{center}
dove le componenti $f_i$, $i = 1,2,3$ si dicono canali.
\\\\Se vogliamo rappresentare una scena in movimento, ottenendo
ciò e un’\textbf{immagine dinamica}, è necessario introdurre una terza
variabile, quella \textbf{temporale} ($t$), per cui si lavora con una funzione
$f: R^3 \rightarrow R^3$.
\begin{center}
    $f(x,y,t) = [f_1(x,y,t), f_2(x ,y,t), f_3(x,y,t)].$
\end{center}
Nelle immagini \textbf{Analogiche} conosco l'intensità di ogni livello di grigio in ogni punto.
Le immagini mostrate al calcolatore invece vanno \textbf{DISCRETIZZATE!}
\section{Discretizzazione}
Se si vuole utilizzare un calcolatore elettronico per lo studio di un
segnale, è necessario \textbf{discretizzare} la funzione $s(t)$ che rappresenta
il segnale. Infatti un calcolatore elettronico è in grado di trattare
solo segnali discreti, cioè successioni di campioni i cui valori sono
rappresentati con precisione finita.
\\Se si lavora con un segnale continuo $s(t)$, per implementarne lo
studio al calcolatore è necessario passare ad un opportuno segnale
discreto.
\begin{center}
    Ciò avviene utilizzando il procedimento di \textbf{campionamento}, che consiste nel discretizzare la variabile temporale $t$.
\end{center}
Inoltre, è anche necessario discretizzare i valori che la funzione $s(t)$
assume (\textbf{quantizzazione}).
\\\\Nel caso delle immagini appliccare i processi di:
\begin{center}
    \textbf{campionamento + quantizzazione}
\end{center}
significa passare da un'immagine \textbf{analogica} ad un'immagine \textbf{digitale}