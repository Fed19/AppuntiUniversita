\chapter{Designing a Faster Flow Algorithm}

Nella sezione precedente, abbiamo visto che qualsiasi modo di scegliere un
augmenting path aumenta il valore del flusso, e questo ha portato a un limite
per C sul numero di augmentations, dove $C = \sum_{e \text{ out of }s} c_e$.
Quando $C$ non è molto grande, questo può essere un limite ragionevole;
tuttavia, è molto debole quando $C$ è grande.\\

L'obiettivo di questo capitolo è mostrare che con una migliore scelta dei
path, possiamo migliorare significativamente questo limite. Una grande
mole di lavoro è stata dedicata alla ricerca di metodi per scegliere
augmenting path nel problema del flusso massimo in modo da minimizzare
il numero di iterazioni. \textbf{Ricordiamo che l'augmentation aumenta
  il valore del flusso del percorso selezionato di un valore che è dato
  dal bottleneck; quindi, è un buon approccio quello di scegliere percorsi
  con una grande capacità di bottleneck.}

\begin{myblockquote}
  \textbf{L'approccio migliore è quello di selezionare il percorso che ha
    il bottleneck di maggiore capacità.}
\end{myblockquote}

Tuttavia, trovare tali percorsi può rallentare di parecchio ogni singola
iterazione. Eviteremo questo rallentamento non preoccupandoci di
selezionare il percorso che ha esattamente la maggiore capacità di
bottleneck. Invece, manterremo un cosiddetto \textbf{scaling parameter}
$\Delta$ e cercheremo percorsi che abbiano un bottleneck di capacità
di almeno $\Delta$. Sia $G_f(\Delta)$ il sottoinsieme del grafo
residuo costituito solo da archi con capacità residua di almeno
$\Delta$. Lavoreremo con valori di $\Delta$ che sono potenze di 2.\\

L'algoritmo è il seguente.

\section{Scaling Max-Flow}


\begin{lstlisting}[language=Python, mathescape=true]
Initially f (e) = 0 for all e in G
Initially set $\Delta$ to be the largest power of 2 that is no larger than the maximum capacity out of s: $\Delta$ $\leq$ max(e out of s ce)

While $\Delta$ $\geq$ 1
  While there is an s-t path in the graph Gf($\Delta$)
    Let P be a simple s-t path in Gf($\Delta$)
    f$'$ = augment(f , P)
    Update f to be f$'$ and update Gf ($\Delta$)
  Endwhile
  $\Delta$ = $\Delta$/2
Endwhile

Return f
\end{lstlisting}

\subsection{Analyzing the Algorithm}

Innanzitutto dobbiamo osservare che l'algoritmo
\texttt{Scaling\ Max-Flow} \textbf{è in realtà solo una variante
  dell'originale algoritmo di Ford-Fulkerson}. I nuovi cicli, il valore
$\Delta$ e il grafo residuo ristretto $G_f(\Delta)$ vengono
utilizzati solo per \textbf{guidare la selezione del percorso residuo,
  con l'obiettivo di utilizzare archi con una grande capacità residua il
  più a lungo possibile.} Inoltre, tutte le proprietà che abbiamo
dimostrato sull'algoritmo \texttt{Max-Flow} originale, sono vere anche
per questa nuova versione: il flusso rimane di valore intero per tutto
l'algoritmo, e quindi tutte le capacità residue sono di valore intero.

\begin{myblockquote}
  \begin{minipage}{\textwidth}
    \begin{definition}\label{def:7.14}
      Se tutte le capacità nella rete di flusso sono intere, allora
      esiste un flusso massimo $f$ per il quale ogni valore di flusso
      $f(e)$ è un numero intero.
    \end{definition}
  \end{minipage}
\end{myblockquote}
\begin{myblockquote}
  \begin{minipage}{\textwidth}
    \begin{definition}\label{def:7.15}
      Se le capacità hanno valori interi, allora in tutto l'algoritmo
      \texttt{Scaling\ Max-Flow} il flusso e le capacità residue rimangono
      valori interi. Ciò implica che quando $\Delta$ = 1, $G_f(\Delta)$ è
      uguale a $G_f$, e quindi quando l'algoritmo termina, $f$ è di valore
      massimo.
    \end{definition}
  \end{minipage}
\end{myblockquote}

\subsection{Costo}

Chiamiamo un'iterazione del ciclo esterno \texttt{While}, con un valore
fisso di $\Delta$, la fase di $\Delta$-scaling. È facile dare un
limite superiore al numero di diverse fasi di $\Delta$-scaling, in
termini di valore di $C = \sum_{e \text{ out of }s} c_e$ che abbiamo
usato anche nella sezione precedente. Il valore iniziale di $\Delta$ è
al massimo $C$, scende di un fattore 2 e non scende mai al di sotto di
1.\\

Quindi: Il numero di iterazioni del ciclo \texttt{While}
esterno è al massimo $\left\lceil 1 + \log_2 C \right\rceil$.\\

La parte più difficile è limitare il numero di \emph{aumenti} eseguiti
in ogni fase di ridimensionamento. L'idea qui è che stiamo usando
percorsi che aumentano molto il flusso, e quindi dovrebbero esserci
relativamente pochi aumenti. Durante la fase di $\Delta$-scaling
utilizziamo solo archi con capacità residua di almeno $\Delta$.\\

Quindi: Durante la fase di $\Delta$-scaling, ogni
augmentation aumenta il valore del flusso di almeno $\Delta$.\\

L'intuizione chiave è che alla fine della fase di $\Delta$-scaling, il
flusso $f$ non può essere troppo lontano dal valore massimo possibile.

\begin{myblockquote}
  \begin{minipage}{\textwidth}
    \begin{theorem}
      Sia $f$ il flusso alla fine della fase di $\Delta$-scaling. Esiste
      un taglio $s-t$ $(A, B)$ in $G$ per cui
      $c(A, B) \le v(f) + m\Delta$ , dove $m$ è il numero di archi nel
      grafo $G$. Di conseguenza, il flusso massimo nella rete ha valore al
      massimo $v(f) + m\Delta$.
    \end{theorem}
  \end{minipage}
\end{myblockquote}

\begin{proof}
  Questa dimostrazione è analoga alla nostra dimostrazione della \ref{def:7.9}, la
  quale stabilisce che il flusso restituito dall'originale
  \texttt{Max-Flow\ Algorithm} è di valore massimo. Come in quella
  dimostrazione, dobbiamo identificare un taglio $(A, B)$ con la
  proprietà desiderata. Sia $A$ l'insieme di tutti i nodi $v$ in $G$
  per i quali esiste un cammino $s-v$ in $G_f(\Delta)$. Sia $B$
  l'insieme di tutti gli altri nodi: $B = V - A$. Possiamo vedere che
  $(A, B)$ è effettivamente un taglio $s-t$ altrimenti la fase non
  sarebbe terminata.\\

  Consideriamo ora un arco $e = (u, v)$ in $G$ per il quale
  $u \in A$ e $v \in B$. Affermiamo che $c_e < f(e) + \Delta$ .
  Infatti, se così non fosse, allora $e$ sarebbe un arco in avanti nel
  grafo $G_f(\Delta)$, e poiché $u \in A$, esiste un cammino $s-u$
  in $G_f(\Delta)$; aggiungendo $e$ a questo cammino, otterremmo un
  cammino $s-v$ in $G_f(\Delta)$, contraddicendo la nostra ipotesi che
  $v \in B$. Allo stesso modo, affermiamo che per ogni arco
  $e' = (u' , v')$ in $G$ per cui $u' \in B$ e $v' \in A$, abbiamo
  $f(e') < \Delta$. Infatti, se $f(e') \ge \Delta$, allora $e'$
  darebbe luogo ad un arco all'indietro $e'' = (v'' , u'')$ nel grafo
  $G_f(\Delta)$, e poiché $v' \in A$, esiste un cammino $s-v'$ in
  $G_f(\Delta)$; aggiungendo $e''$ a questo cammino, otterremmo un
  cammino $s-u'$ $G_f(\Delta)$, contraddicendo la nostra ipotesi che
  $u' \in B$.\\

  Quindi tutti gli archi $e$ uscenti da $A$ sono quasi saturati
  (soddisfano $c_e < f(e) + \Delta$) e tutti gli archi entranti in $A$
  sono quasi vuoti (soddisfano $f(e) < \Delta$).\\

  Possiamo ora usare \ref{def:7.6} per raggiungere la conclusione desiderata:

  $$
    v(f) = \sum_{e \text{ out of } A}f(e) - \sum_{e \text{ into } A}f(e) \ge \sum_{e \text{ out of } A}(c_e - \Delta) - \sum_{e \text{ into } A}\Delta =
  $$
  $$
    \sum_{e \text{ out of } A}c_e - \sum_{e \text{ out of } A}\Delta - \sum_{e \text{ into } A}\Delta \ge c(A, B) - m\Delta
  $$
\end{proof}

Qui la prima disuguaglianza segue dai nostri limiti sui valori di flusso
degli archi attraverso il taglio, e la seconda disuguaglianza segue dal
semplice fatto che il grafo contiene solo $m$ archi in totale. Il
valore del flusso massimo è limitato dalla capacità di qualsiasi taglio
di \ref{def:7.8}. Usiamo il taglio $(A, B)$ per ottenere il limite dichiarato
nella seconda affermazione.

\begin{myblockquote}
  \begin{minipage}{\textwidth}
    \begin{definition}\label{def:7.19}
      Il numero di aumenti in una fase di ridimensionamento (\textbf{scaling})
      è al massimo di $2m$.
    \end{definition}
  \end{minipage}
\end{myblockquote}

\begin{proof}
  L'affermazione è chiaramente vera nella prima fase di scaling: possiamo
  usare ciascuno degli archi di $s$ solo per al massimo un augmentation
  in quella fase. Consideriamo ora una successiva fase di scaling
  $\Delta$, e sia $f_p$ il flusso alla fine della precedente fase di
  scalatura. In quella fase, abbiamo usato $\Delta' = 2\Delta$ come
  nostro parametro. Per la (7.18), il flusso massimo $f^*$ è al massimo
  $v(f^*) \le v(f_p) + m\Delta' = v(f_p) + 2m\Delta$ . Nella fase di
  $\Delta$-scalatura, ogni augmentation aumenta il flusso di almeno
  $\Delta$ , e quindi possono esserci al massimo $2m$ augmentations.
\end{proof}

Una augmentation richiede un tempo $O(m)$, compreso il tempo
necessario per impostare il grafo e trovare il percorso appropriato.
Abbiamo al massimo $1 + \left\lceil log_2 C \right\rceil$ fasi di
ridimensionamento $C$ e al massimo $2m$ augmentations in ciascuna
fase di ridimensionamento. Abbiamo quindi il seguente risultato.

\begin{myblockquote}
  \begin{minipage}{\textwidth}
    \begin{theorem}\label{th:7.20}
      L'algoritmo \texttt{Scaling\ Max-Flow} in un grafo con $m$ archi e
      capacità intere trova un flusso massimo in al massimo
      $2m(1 + \left\lceil log_2 C \right\rceil)$ augmentations. Può essere
      implementato per eseguire al massimo in tempo $O(m^2 \cdot log_2 C)$.
    \end{theorem}
  \end{minipage}
\end{myblockquote}

Quando $C$ è grande, questo limite temporale è molto migliore del
limite $O(mC)$ applicato a un'implementazione arbitraria
dell'algoritmo di \texttt{Ford-Fulkerson}. Il generico algoritmo di
Ford-Fulkerson richiede un tempo proporzionale alla grandezza delle
capacità, mentre l'algoritmo di scaling richiede solo un tempo
proporzionale al numero di bit necessari per specificare le capacità
nell'input del problema . Di conseguenza, l'algoritmo di
ridimensionamento funziona in tempo polinomiale nella dimensione
dell'input (ovvero, il numero di archi e la rappresentazione numerica
delle capacità), e quindi soddisfa il nostro obiettivo tradizionale di
ottenere un algoritmo polinomiale.
