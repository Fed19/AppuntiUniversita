\newpage

\section{Trust Network Brainstorming}

Ogni sistema di sicurezza dipende dalla fiducia, in una forma o nell'altra,
degli utenti del sistema. In
generale esistono diverse forme di fiducia per affrontare diversi tipi di problemi
e mitigare il rischio
in determinate condizioni. Quale forma di fiducia applicare in una determinata
circostanza è generalmente dettata dalla politica aziendale.
Le \textbf{trust network} (reti di fiducia) consistono in ramificazioni di connessioni interpersonali costituite principalmente da forti legami in cui una parte è disposta a dipendere da qualcuno, in una determinata situazione, con una sensazione di relativa sicurezza, anche se sono possibili conseguenze negative a causa del fallimento altrui.
Le relazioni di fiducia consentono agli utenti di un dominio di accedere alle
risorse di un altro dominio. I trust funzionano facendo sì che un dominio si
affidi all'autorità dell'altro dominio per
l'autenticazione dei suoi account utente.
La scelta che deve affrontare l'amministratore della sicurezza consiste nel
dare maggiore importanza all'autonomia o alla sicurezza del sistema:
in entrambi i casi, l'altro aspetto verrà necessariamente sacrificato.

Le proprietà del trust sono:

\begin{itemize}
    \item grafo in cui i nodi sono gli utenti mentre gli archi indicano la
          relazione di fiducia;
    \item peso associato ad ogni arco (\verb|0 = mistrust|, \verb|1 = max trust|);
    \item soggettività: l'utente può avere opinioni differenti rispetto agli altri;
    \item asimmetria: la fiducia che detiene un utente verso un altro non
          deve essere
          necessariamente speculare;
    \item i giudizi sono dipendenti dal contesto.
\end{itemize}

\paragraph{Semirings}
viene utilizzata come struttura generica per il calcolo della
fiducia nelle reti fiduciarie (cioè pesate). I \textit{semirings} ``bipolari''
permettono di considerare insieme fiducia e sfiducia e di avere
così una composizione sicura delle reti trust
(avviene una fusione critica ogni volta che due comunità basate sulla fiducia
devono essere amalgamate).