\section{Sicurezza e Blockchain}

\subsection{Double Spending}

È un potenziale attacco in cui un bitcoin viene inviato a 2 persone
contemporaneamente, benché non sia realizzabile nella blockchain di bitcoin
poiché è intrinsecamente resistente fa da base per molti altri attacchi e dunque
spiegheremmo brevemente come funziona. Un miner A invia denaro a due utenti (negozi) B e C contemporaneamente, di norma in bitcoin viene approvata la prima transazione
(dato che ci si basa sull'hash e sul timestamp) ma supponiamo che entrambe vengano approvate in contemporanea su due blocchi diversi, allora entrambe verranno viste come transazioni valide e ci sarà una possibile fork della
blockchain. In bitcoin però prima o poi una delle due blockchain risulterà essere più lunga dell'altra e verrà presa quella come blockchain valida e perciò una delle due transazioni verrà rifiutata ma nel mentre il miner ha già ricevuto gli oggetti comprati sui negozi B e C. La soluzione a tale problema è che i negozianti, prima di inviare le proprie merci, devono aspettare che la propria transazione risulti assere aggiunta in maniera definitiva alla blockchain e ciò avviene dopo il mining di altri 6 blocchi rispetto a quello dove è tale transazione (circa 1 ora di attesa).

\subsubsection{51\% Attack}

Un possibile attacco che concettualmente è simile a quello del double spending è
quello del 51\% in cui un utente che dispone di un ampia porzione di potenza di
calcolo della rete può tentare di manipolare la blockchain portando avanti in
segreto una copia alterata della stessa. Se il miner che sta portando avanti
questo attacco riesce a generare una blockchain più lunga di quella vera allora
per la regola della catena più lunga, divulgandola la farà accettare dalla rete
e diventerà dunque la vera blockchain.\\
Per imporre i suoi blocchi, un'organizzazione dovrebbe controllare il 51\% dei
nodi (è detto 51\% attack), per avere potenza di calcolo sufficiente a produrre
(con buona probabilità) catene di blocchi lunghe prima di tutti gli altri nodi,
ciò è altamente improbabile. Attualmente non è possibile che questo attacco abbia successo perché nessuno dispone di una potenza di calcolo così elevata ma ci sono comunque delle \emph{mining pool} che accrescono quotidianamente la propria hash power (la più grande ha circa il 30\%).

\subsection{Wallet attack}

Ti vengono rubate le credenziali di accesso al wallet o le chiavi private
\emoji{man-dark-skin-tone}.

\subsection{Transaction Malleability }

Gli attacchi di malleability consistono nello sfruttare la manipolabilità
dell'id delle transazioni che viene calcolato con un hash. Andando a cambiare
l'id della transazione si verificherà un effetto a cascata causato dall'hash e
dunque il mittente della transazione non la riconoscerà più come propria anche
se c'è la possibilità che i soldi vengano comunque inviati. Cosi facendo
l'utente malevolo ha un pretesto per richiedere nuovamente i soldi alla vittima
affermando che non ha mai inviato i soldi.

\subsection{Sybil Attack}

Un Sybil attack è un attacco in cui un utente o un gruppo di utenti prende il
controllo della rete creando un elevato numero di nuovi account e nodi, avendo
così la maggioranza dei nodi. Questo consentirebbe di poter far approvare dei
blocchi con transazioni fraudolente. Tutto ciò risulta essere impossibile dato
che viene mitigato dalla PoW.