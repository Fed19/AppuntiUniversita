\section{C.I.A.}

Quando si parla di protezione e sicurezza dei dati informatici,
esistono \textbf{tre principi fondamentali} su cui focalizzare l'attenzione,
in nome di una corretta gestione: vale a
dire \textbf{Confidentiality} (\textit{Riservatezza}),
\textbf{Integrity} (\textit{Integrità}) e
\textbf{Availability} (\textit{Disponibilità})
(detta anche ``\textit{Triade C.I.A.}'').
Tali principi devono essere ricercati in ogni soluzione di sicurezza, tenendo
conto anche delle implicazioni introdotte dalle vulnerabilità e dai rischi.
Per questo motivo vengono detti ``componenti base'':

\paragraph{Riservatezza dei Dati:}

Una strategia volta alla privacy informatica deve in prima battuta offrire
riservatezza, ovvero garantire che i dati e le risorse siano preservati dal
possibile utilizzo o accesso da parte di soggetti non autorizzati.
Ciò deve valere per tutte le fasi di vita del dato.
Le cause di violazione della riservatezza possono essere imputabili ad un attacco
malevolo oppure ad un errore umano. Le modalità di attacco possono essere molteplici:
si va per esempio dalla sottrazione di password all’intercettazione di dati su
una rete. Meccanismi per garantire la riservatezza possono essere:

\begin{itemize}
    \item controllare l'accesso ai dati cifrandoli:
          solo chi ha la chiave riesce ad accedervi;
    \item meccanismi dipendenti dal sistema (sono a più basso livello).
\end{itemize}

Se viene usato un certo meccanismo, significa che si sono fatte delle assunzioni
di affidabilità, quindi si presuppone che esso sia funzionante e perciò posso fidarmi.
Si potrebbe argomentare, in realtà, che la riservatezza debba anche affrontare,
non solo la necessità di voler nascondere il contenuto di un documento da occhi
indesiderati, ma anche la sua esistenza. Se consideriamo, ad esempio, l’analisi
di un sistema di comunicazione, potremmo certamente non essere in grado di leggere
i messaggi tra due utenti, ma allo stesso tempo potremmo derivare informazioni
importanti in base alla frequenza di scambio di messaggi.
Pensiamo, ad esempio, alla frequenza di messaggi scambiati da un Server bancario
rispetto a quella di un semplice utente.
La definizione di riservatezza dovrebbe essere in grado quindi di coprire un
altro aspetto, chiamato \textit{Unlinkability}: due o più oggetti di interesse
(messaggi, azioni, eventi, utenti) non sono correlabili se un attaccante non è
in grado di distinguere la loro relazione.
Non di meno, la definizione di riservatezza dovrebbe includere anche la
possibilità che un utente non voglia poter essere riconosciuto. Vogliamo cioè
l'\textit{Anonimato}: un utente è Anonimo se non può essere identificato in
un insieme di soggetti anonimi.

\paragraph{Integrità:}
intesa come capacità di mantenere la veridicità dei dati e delle risorse e
garantire che non siano in alcun modo modificati o cancellati, se non ad opera
di soggetti autorizzati. Consiste quindi nella prevenzione verso modifiche
improprie o non autorizzate dei dati, al fine di garantire una fiducia sugli stessi.
Per assicurare l’integrità è necessario mettere in atto policy di autenticazione
chiare e monitorare costantemente l’effettivo accesso ed utilizzo delle risorse,
con strumenti in grado di creare log di controllo. Le violazioni all’integrità
dei dati possono avvenire a diversi livelli.
Ad essa è spesso affiancata l'autenticazione: infatti, oltre a concentrarsi sui
cambiamenti del contenuto, occorre salvaguardarsi anche dalle modifiche che
possono cambiare l’origine dei dati. Si può quindi spesso effettuare un attacco
sia di autenticazione che di integrità: per esempio, quando in un messaggio viene
cambiato il mittente.
I possibili meccanismi per garantire l'integrità sono:

\begin{itemize}
    \item la \textbf{prevenzione}: che consiste nel vietare le modifiche a
          chiunque o ad utenti
          specificatamente non autorizzati (ciò è però difficile da ottenere);
    \item la \textbf{scoperta}: cioè accorgersi che ci sono state modifiche
          nell'origine o nel contenuto dei dati
          e reagire in modo opportuno.
\end{itemize}

\paragraph{Disponibilità:}
si riferisce alla possibilità, per i soggetti autorizzati, di poter accedere
alle risorse di cui hanno bisogno per un tempo stabilito ed in modo ininterrotto.
Rendere un servizio disponibile significa anche garantire che le risorse
infrastrutturali siano pronte per la corretta erogazione di quanto richiesto.
Devono quindi essere messi in atto meccanismi in grado di mantenere i livelli di
servizio definiti.
È più difficile effettuare un attacco alla disponibilità della risorsa piuttosto
che alla sua affidabilità (reliability). Va anche detto che la disponibilità è
più facile da garantire rispetto all'affidabilità.
I possibili attacchi in questo campo sono per esempio:

\begin{itemize}
    \item la manipolazione dell'utilizzo di dati e risorse;
    \item Denial of Service (DoS).
\end{itemize}